\documentclass[12pt]{exam}
\usepackage[T1]{fontenc}
\usepackage[utf8]{inputenc}
\usepackage[brazil]{babel}
\usepackage{lmodern}
\usepackage{graphicx}
\usepackage[normalem]{ulem}
\usepackage{hyperref}

\footer{}{}{}

\title{}
\date{}

\newcommand{\siheader}{
\begin{center}
\includegraphics{archi.png}
\vspace{0.5cm}
{\huge Prova de Estágio na Seção de Informática}
\vspace{0.3cm}
\makebox[\textwidth]{Nome:\enspace\hrulefill}
\makebox[\textwidth]{Assinatura:\enspace\hrulefill}
\end{center}
}


\begin{document}
\siheader
\section*{Recomendações}
Para as seções posteriores, se necessário, use:

\subsection*{Credenciais}
\begin{verbatim}
usuário: zelele
senha: zelele123
\end{verbatim}

%%% HARDWARE %%%
\section*{Hardware}
\begin{questions}

\question
Montar o computador.

\begin{parts}
\part
Conecte adequadamente as placas.

\part
Conecte adequadamente os periféricos.

\part
Ligue o computador.
\end{parts}

\question
Carregar o sistema Windows já instalado.

\end{questions}
\newpage


%%% WINDOWS %%%
\section*{Windows}
\begin{questions}

\question
Configurar corretamente o teclado.

\question
Verificar o endereço \verb+IP+ do computador e mudar.
\begin{itemize}
    \item Novo IP: \verb+192.168.57.3+
    \item Máscara: \verb+255.255.192.0+
    \item Gateway: \verb+192.168.45.1+
    \item DNS: \verb+143.107.45.1+
\end{itemize}

\question
Instalar a impressora \verb+Escada+ e imprimir uma página de teste. O
endereço dela na rede é:

\begin{quote}
\verb+http://cups.ime.usp.br:631/printers/Escada+
\end{quote}

e o \textit{driver universal} que ela usa está em:
\begin{quote}
    \url{https://si.ime.usp.br/downloads/SamsungUniversalPrintDriver3.exe}.
\end{quote}

\question
Dadas as credenciais:
\begin{itemize}
\item usuário: \verb+zeprova+
\item senha: \verb+pratiqueatividadesfisicasregularmente+
\end{itemize}

Mapear a seguinte pasta de rede:
\begin{quotation}
    \verb+\\DONBOT\estagiprova+
\end{quotation}

\question
Instalar o gerenciador de pacotes \textit{Chocolatey} e, a partir dele, o pacote \verb+adobereader+.

\question
Instalar o \textit{R}.

\question
Criar o usuário \verb+zemane+ com a senha \verb+zemane123+.

\question
Salve na Área de Trabalho uma lista dos itens de inicialização do Windows. Pode ser texto, imagem ou qualquer formato legível.

\end{questions}
\newpage


%%% LINUX %%%
\section*{Linux}
\begin{questions}

\question
Verificar o endereço \verb+IP+ do computador e mudar para \verb+192.168.57.3+,
com a máscara \verb+255.255.192.0+ e o \verb+gateway+ como \verb+192.168.45.1+
 e \verb+DNS+ como \verb+143.107.45.1+.

\question
Instalar a impressora \verb+Escada+ utillizando \textit{Generic IPP Everywhere} e imprimir uma página de teste. O endereço dela na rede é:

\begin{quote}
\verb+http://cups.ime.usp.br:631/printers/Escada+
\end{quote}

\question
Instalar a versão atualizada do navegador \textit{Chromium}.

\question
Instalar o \textit{Libre Office} utilizando o repositório \textit{backports} da versão correspondente à distribuição instalada.

\question
Instalar o \verb+PlayOnLinux+ utilizando algum gerenciador de pacotes.
\begin{parts}
\part
Adicionar a arquitetura \verb+i386+ usando os comandos:
\begin{itemize}
    \item{\verb+dpkg --add-architecture i386+}
    \item{\verb+apt update+}
\end{itemize}
\part
Usando o gerenciador de pacotes, instalar \verb+wine32-development+.

\part
Dentro do \verb+PlayOnLinux+, instalar o \textit{Notepad Plus Plus}.
\end{parts}

\question
Criar o usuário \verb+zemane+ com a senha \verb+zemane123+.

\question
Logar via \verb+SSH+ no \verb+IP 192.168.57.101+ com o usuário \verb+estag4+
e a senha
\begin{quotation}
    \verb+pratiqueatividadesfisicasregularmente+
\end{quotation}
e criar um arquivo de texto contendo o nome do processo que está consumindo a maior quantidade de CPU.

\question
Reconfigurar o \textit{swap}.
\begin{parts}
\part
Identificar qual partição está utilizando o \textit{swap}.
\part
Desligar o \textit{swap};
\part
Criar um novo \textit{swap} a partir de um arquivo\footnote{Pode ser interessante fazer antes o exercício seguinte.}.
\end{parts}

\question
A partir do LVM:
\begin{parts}
\part
Criar um volume físico com a partição que continha o \textit{swap};
\part
Criar um volume lógico a partir do volume físico;
\part
Formatar o volume lógico como \verb+ext4+;
\part
Montá-lo em \verb+/tmp+.
\end{parts}

\end{questions}
\end{document}
